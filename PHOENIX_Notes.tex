\title{Notes on PHOENIX Spectra:}
\author{
        Jason Neal \\
        IA, Portugal\\
        }
\date{\today}

\documentclass[12pt, a4paper]{article}
\usepackage{ifpdf}
\ifpdf
\usepackage[breaklinks,hidelinks]{hyperref}
\else
\usepackage{url}
\fi
\usepackage{listings}   
\usepackage[textwidth=16cm,]{geometry}

\begin{document}
\maketitle

\begin{abstract}
A cheat sheet for phoenix spectra.
\end{abstract}

\section{Introduction}

This is my general understanding of the PHOENIX stellar atmosphere models, I do not claim it is complete or  100\% correct. The PHOENIX modelling code has evolved overtime incorporating new physical models to better explain the atmospheres. These also usually come with the inclusion of new/improved lines of atmospheric species, or using the newest solar abundances, e.g. Aspund 2009, or CIFIST --- (BT2 line list BT lines. Barber et al 2006 A high-accuracy computed water line list)

The PHOENIX code started life as modelling code for Novae type objects. It was then "Extended (1995)" to be used for modelling the atmospheres of Low Mass Stars(LMS) / Brown Dwarfs(BDs). 
The NEXTGEN models (1996-1999) (e.g. Allard et al .1999) were the next iteration that treats the atmosphere as a gas in chemical equilibrium,  but poorly model the spectra for very low mas stars as no treatment of dust.

The next iteration was the Dusty/Cond models (Allard et al 2001) which investigate both extreme limits of clouds. They include the physics of condensation, in which the gas is allowed to condense (Gibbs free energy, partial pressures of gases etc.) into the Chemical equilibrium (without any species being removed). They also add a interaction of the condensation/dust with the light (add opacities of the dust, scattering). For the \textbf{Dusty model} the condensation/dust stays in the atmosphere where is created \textbf{(inefficient/no settling}) and it effects the spectrum through the dust opacities.  For the \textbf{Cond models} however the dust opacities are set to zero, to simulate \textbf{Efficient settling} in which all the clouds fall below the spectrum forming region. These also use a NASA AMES line list which is why they are called AMES-Dusty/AMES-Cond. Future models are re-run with the BT line list and a called BT-Cond/BT-Dusty.
These to situations represent the limiting cases and BDS should transition from Dusty -> Cond was their temperatures get cooler (a have more settling), real LMS/BDs should fall between these to models somewhere. 

Above 2600K the Dusty/COnd models are similar. This is because of the phase change due to crystallization of silicates in atmospheres below Teff= 2600K.

The next iteration \textbf{AMES-Settl} starts to include the movements of dust in the atmosphere. Add limited diffusion model and removal the condensed particles from the chemical equilibrium, also include convective upwelling. (Allard et al 2003)

The over the next 10 years they add supersaturation and 2D radiation hydro-dynamical modeling for the diffusion and mixing. 
They use the BT2 water opacity line list Barber et al .2006, hence BT-Settl name change, and the Asplund et al 2009 solar abundances, which improve the results significantly.
They find that gravity waves also play an important role in BD cloud formation and The BT-Settl also gives "perfectly fitting spectra distribution across the nIR- IR spectral region". (Allard et al 2010)  


The newest BT-Settl models presented in Barrafe et al. 2015  add a \textbf{consistent interior} and also \textbf{evolution} (age). These are the most recently publish BT-Settl models available and use the Aspund 2099 abundances with some revisions by \textbf{ Caffau et al. (2011) solar abundances}.


Phoenix ACES:
The phoenix ACES model are quite different from the BT-Settl models. They start from the Cond models criteria and limit themselves to >2300K to avoid dealing with clouds. The switch out the chemical equation solver to the Astrophysical   Chemical   Equilibrium   Solver   (ACES, Barman 2012). They \textbf{do not comment on how this new solver affect the abundances of species}.


Parametrisations, for the mass and the mixing length parameter \(\alpha\) to better match observations. Usually the mixing length has been fixed or solved using the 2D radiation hydro-dynamical modeling.
3D radiative hydrodynamic models for convection mixing length parameter but \textbf{do not comment on how this affects the spectra} due to all the other changes.  
Astrophysical   Chemical   Equilibrium   Solver   (ACES,
Barman 2012) 

They also use the Aspuland 2009 solar abundances.


In practice the ACES spectra seem too many lines that don't seem to match the stellar spectra.   Seem to be sharper/deeper then other models.


\paragraph{Outline}

Summary some of the different models

\begin{table}
	\begin{tabular}{lccccc}
		\hline
		Model & year & Phoenix version  & Teff Range &  Notes &Reference\\
		\hline
		PHOENIX-ACES & 2013 & 16 & 2300-12000K & new ACES EOS & \href{https://arxiv.org/abs/1303.5632}{Husser 2013} \\
		
    	Settl & 2003+&15 & & & \href{}{} \\
    	Cond & 2001 & & & & \href{}{} \\
    	Dusty & 2001& & & & \href{}{} \\
        NEXTGEN & 1996-1999& & & \href{}{} \\
        EXTENDED &1995 & & & & \href{}{} \\
    \hline
	\end{tabular}
\end{table}



\begin{table}
    \begin{tabular}{lccccc}
        \hline
        Model & Phoenix version & Line Lists & Teff Range &  Notes &Reference\\
        \hline
        PHOENIX-ACES & 16 & & 2300-12000K & new EOS & \href{https://arxiv.org/abs/1303.5632}{Husser 2013} \\
        
        BT-Settl & 15 & & & & \href{}{} \\
        BT-Cond & & & & & \href{}{} \\
        BT-Dusty & & & & & \href{}{} \\
        AMES-Cond & & & & & \href{}{} \\
        AMES-Dusty & & & & & \href{}{} \\
        \hline
    \end{tabular}
\end{table}


\begin{table}
    \begin{tabular}{lcc}
        \hline
        Model & Line Lists&Reference\\
        \hline
        AMES & NASA AMES H2O and TiO line list&  \href{}{ } \\
        
        BT2& High resolution H20 & \href{}{Barber et al. 2006} \\
        Solar &  & \href{}{} \\
        Solar & & \href{}{Caffau et al. (2011)} \\
        Solar &  & \href{}{Aspund 2009} \\
        
        \hline
    \end{tabular}
\end{table}


\section{Previous work}\label{previous work}


\subsection{PHOENIX ACES}:
Available from Goettingen  \href{http://phoenix.astro.physik.uni-goettingen.de/}{http://phoenix.astro.physik.uni-goettingen.de/}
The PHOENIX-ACES library can be hound at...
The spectra are in fits format and have headers etc. 

PHOENIX 16 code.
Quote:
Many changes have been implemented with respect to previous
PHOENIX models. For example, the more detailed treatment of
the chemical equilibrium in  the  new EOS strongly affects  the
stellar structure and results in different line and molecular band
strengths, which can introduce significant differences in comparison to older PHOENIX model spectra, especially for M stars as
discussed below. But major changes can also be expected from
the new list of element abundances and the new parameterizations
for the mixing length and the micro-turbulence. Consequentially
we observed some significant differences between spectra from
previous PHOENIX grids and from this one.
We  compared some temperature profiles with those of the
original models that have been used as a starting point for the
new library. For \(\tau> 1\) they match very well and we only see
differences for \(\tau> 1\), which are irrelevant for the morphology
of the final spectrum


Doubly affected by new atmosphere model(v16) and new physics in grid. 

Resolution = 

\section{BT Models}

 In the case of the most recent models, the Barber \& Tennison (UCL) so-called \textbf{ BT2 water vapor line list} has been used explaining why all those models bear names starting with 'BT-'. \textbf{BT-Dusty refers to dust in equilibrium with the gas phase }(sedimentation is
 neglected), while \textbf{BT-Cond includes dust condensation in equilibrium with the gas phase while neglecting their opacities in the radiative transfer}. \textbf{BT-Settl means that gravitational settling of sedimentation is accounted for in the frame of a detailed cloud model (slightly adapted from Rossow '78) which also account for supersaturation, nucleation, sedimentation and mixing.}  


\subsection{BT-Settl}

The following section is adapted straight from \url{https://phoenix.ens-lyon.fr/Grids/FORMAT}, tidied up with some extra useful notes added.

\subsubsection{FORMAT OF THE SPECTRA OUTPUT FILES}
These are the Spectra of France Allard

You can find the  pre-computed grids at \href{}{\url{add link here}}, also accessible via links on
the bottom part of the simulator presentation page, or using this link:
\href{http://phoenix.ens-lyon.fr/Grids/}{\url{http://phoenix.ens-lyon.fr/Grids/}}

The file names contain the main parameters of the models\\
lte{\(T_{eff}\)/10}-{Logg}{[M/H]}a[alpha/H].GRIDNAME.7.spec.gz/bz2/xz\\
is the synthetic spectrum for the requested effective temperature
(\(T_{eff}\)),surface gravity (Logg), metallicity by log10 number density with
respect to solar values ([M/H]), and alpha element enhancement relative     
to solar values [alpha/H]. The model grid is also mentioned in the name.

Spectra are provided in an ASCII format (*.7.gz):\\
	column1: wavelength in Angstrom\\
	column2: 10**(F\_lam + DF) to convert to Ergs/sec/\({cm}^2\)/A\\
	column3: 10**(B\_lam + DF) i.e. the blackbody fluxes of same \(T_{eff}\) in same units.\\


DF= -8.d0 for all most recent models (Ergs/sec/\({cm}^2\)/cm). For older model
series like the NextGen and AMES-Cond grids DF= -26.9007901434d0, 
because previous Phoenix outputs were giving out the luminosity, 
L (= \(R^2\) * H) in erg/s/\({cm}^2\)/cm.  And for NextGen spectra
of effective temperature 5000K and above, DF'= -28.9007901434d0.

Additional columns, obtained systematically when computing spectra using the
Phoenix simulator, give the information to identify atomic and molecular
lines. This information is used by the idl scripts lineid.pro and plotid.pro 
which are provided in the user result package.  

\textbf{Note:}
- PyAstronomy's \href{www.hs.uni-hamburg.de/DE/Ins/Per/Czesla/PyA/PyA/pyaslDoc/phoenixUtils/phoenixUtils.html}{PhoenixUtils.readUnit7} function is able to read the newer file version directly into Ergs/sec/\({cm}^2\)/cm. Multiply by 10**-8 to convert to Ergs/sec/\({cm}^2\)/A
- If handling the files manually in Python they are in IDl double format (e.g. 1.7234D+1) you need to string replace the exponential D with E to be able to convert into python floats (PyAstronomy does this already)


With the stacked ASCII format (*.spec.gz files ) we have rather:
\[
line1: T_{eff} logg [M/H] of the model
line2: number of wavelengths
line3: F_lam(n) X 10**DF , n=1,number of wavelengths
lineX: B_lam(n) X 10**DF , n=1,number of wavelengths
\]
This older file format is no longer used and no ID output is provided with those files.\\


\textbf{A very important point is that since models are often computed on parallel
computers using several nodes, it is important to sort the spectra files in
increasing wavelength order prior to using them.}

Please note that a conversion of the fluxes to absolute fluxes as measured at
the earth requires a multiplication by the dilution factor
$(radius/distance)^2$. The distance cancels out when accounting 
simultaneously for the dilution factor and distance modulus at 10 pc
(for absolute magnitudes for instance). This is done using the following
formula:
\[\rm M = m' - 5 * log10 ( R_{*} [in\ pc] ) + 5, \]

where m' is the magnitude associated to the flux F' (= 10**(lgF + DF), 
where F is the value directly contained in the spectrum file). 

\textbf{The model parameters are indicated by the name of the files.}
E.g. ``lte030-6.0-0.0.AMES-dusty.7.gz'' means a model with \(T_{eff}\)=3000K; 
logg=6.0 and [M/H]=0.0, and that it is computed using both the 
TiO and H2O line lists from Nasa \textbf{AMES}, and that it includes full 
\textbf{dust }treatment (both condensation AND opacities). In the case of the most
recent models, the Barber \& Tennison (UCL) so-called \textbf{ BT2 water vapor line list}
has been used explaining why all those models bear names starting with 'BT-'. 
\textbf{BT-Dusty refers to dust in equilibrium with the gas phase }(sedimentation is
neglected), while \textbf{BT-Cond includes dust condensation in equilibrium with the
gas phase while neglecting their opacities in the radiative transfer}. \textbf{BT-Settl
means that gravitational settling of sedimentation is accounted for in the
frame of a detailed cloud model (slightly adapted from Rossow '78) which also
account for supersaturation, nucleation, sedimentation and mixing.}   

Note that Phoenix delivers synthetic spectra in the vacuum and that a line
shift is necessary to adapt these synthetic spectra for comparisons to
observations from the ground. For this, divide the vacuum wavelengths by
(1+1.e-6*nrefrac) as returned from the function below to get the air 
wavelengths (or use the equation for AIR from it). 

\lstset{language=Pascal} 
\begin{lstlisting}[language=python, frame=single]  % Start your code-block

def nrefrac(wavelength, density=1.0):
    """Calculate refractive index of air from Cauchy formula.

    Input: wavelength in Angstrom, density of air in amagat (relative to STP,
    e.g. ~10% decrease per 1000m above sea level).
    Returns N = (n-1) * 1.e6. 
    """

    # The IAU standard for conversion from air to vacuum wavelengths is given
    # in Morton (1991, ApJS, 77, 119). For vacuum wavelengths (VAC) in
    # Angstroms, convert to air wavelength (AIR) via: 

    #  AIR = VAC / (1.0 + 2.735182E-4 + 131.4182 / VAC^2 + 2.76249E8 / VAC^4)

    try:
        if isinstance(wavelength, types.ObjectType):
            wl = np.array(wavelength)
    except TypeError:
        return None

    wl2inv = (1.e4 / wl)**2
    refracstp = 272.643 + 1.2288 * wl2inv  + 3.555e-2 * wl2inv**2
    return density * refracstp
\end{lstlisting}

\textbf{Note:} \href{http://www.hs.uni-hamburg.de/DE/Ins/Per/Czesla/PyA/PyA/pyaslDoc/aslDoc/pyasl_wvlconv.html}{PyAstronomy}\footnote{\url{http://www.hs.uni-hamburg.de/DE/Ins/Per/Czesla/PyA/PyA/pyaslDoc/aslDoc/pyasl_wvlconv.html}} also has functions to handle air/vacuum conversions. 


\textbf{DISCLAIMER}

\textit{
The model atmospheres and synthetic spectra ARE ONLY addressing the PHOTOSPHERE
and do not include parts of the atmosphere which are not governed by
hydrostatic equilibrium such as chromosphere's and corona for stars or
exospheric evaporation in the case of irradiated planets or stars. }

\textit{The simulator uses the most recent code version to generate as close as
possible spectra compatible with the published grids i.e. by adopting the
same parameters (mixing length, geometry of the radiative transfer, opacities
for the most important when possible). No attempts has been made to reproduce 
those older results exactly.}


\section{Using/Accessing models}

PyAstronomy readUnit7()  for loading ``XXX.7'' files

Starfish  grid\_tools: \\BTSettlGridInterface (older unit7 type),\\ CIFISTGridInterface (cifist2011\_2015 fits files), \\ PHOENIXGridInterfaceNoAlpha (PHOENIX-ACES)


\subsection{Other Models}
Mode models can be found at the 
\href{http://svo2.cab.inta-csic.es/theory/newov/index.php}{Theoretical Spectra Web Server}

\end{document}
